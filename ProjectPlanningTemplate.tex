\newcommand{\comment}[1]{}

%\documentclass[a4paper,twocolumn,12pt]{article}

%\documentclass[a4wide,12pt]{report}

%\documentclass[a4wide,12pt]{article}
%\documentclass[informasjonssikkerhet]{gucmasterproject}
\documentclass[informationsecurity]{gucmasterproject}

%\usepackage{pslatex} %% Doesn't seem to work - i.e. convert .eps to .pdf
 
\usepackage[utf8]{inputenc}     % For utf8 encoded .tex files
%\usepackage[latin1]{inputenc}
\usepackage[norsk]{babel}     % For chapter headings etc.
%\usepackage[pdftex]{graphicx}           % For inclusion of graphics

%From http://math.uib.no/it/tips/
   %% For grafikk
    \usepackage{ifpdf}
    \ifpdf
      \usepackage[pdftex]{graphicx}
      \usepackage{epstopdf}
    \else
      \usepackage[dvips]{graphicx}
    \fi
    %% Her kan du putte dine vanlige pakker og definisjoner



%\usepackage[dvips]{hyperref}    % For cross references in pdf
\usepackage{hyperref}
\usepackage{mdwlist}
\usepackage{url}
\usepackage{here}

\def\UrlFont{\tt}

\begin{document}

\thesistitle{Service Design - NAV}
\thesisauthor{Engedal, J. Ø., Grimsgaard, C., Pedersen, K.}
\thesisdate{\gucmasterthesisdate}
\useyear{2014}
\makefrontpages % make the frontpages
%\thesistitlepage % make the ordinary titlepage


\comment{
Front page - including
"   HIG technical report front page including logos etc.
"   The text: "MSc project plan"
"   Title of project
"   Name of author and contact details
"   Date
"   Version

email address
"   MAIS students must include "NISlab" as their affiliation.
Date:22.10.2003

Structure of MSc thesis project plan
Gj�vik University College
}


\chapter*{Revisjoner}

\begin{center}
\begin{tabular}[H]{|l|p{35em}|}
\hline
Version \#  & Description of change (why, what where - a few sentences)\\
\hline
      0.1   & First version made available via Fronter\\
\hline
      0.2   & Corrected some spelling mistakes and added 'control questions' to Abstract, chapter 1 and 2\\
\hline
      0.2.1   & Removed the reference to a dead link in chapter 1 (keywords).\\
\hline
\end{tabular}
\end{center}
\newpage

\begin{abstract}
Abstract (1/2 page)
This document provides format and guidelines  for the 
MSc project descriptions. The document has been produced using MikTeX and TeXnicCenter.

The objective of the abstract is to provide the reader with an understanding of the work to be done and put him in the position to make a 'correct' decision regarding  reading/not reading the report.

The abstract of the project description
{\em must} include
\begin{itemize}
\item a summary of the problem description,
\item motivation and 
\item a summary of the planned contribution from the master project in terms of {\em new} results.
\end{itemize}

\paragraph{Control questions}
\begin{enumerate}
\item Does the abstract have a 'reasonable' length?
\item Is it clear to a non expert (e.g. a typical reader of a newspaper) what problem is addressed?
\item Does a person that has been working in the field find the text informative?
\item Do the results that might be obtained have the potential to be interesting to a lot of people? How interesting to how many and why?
\item Would a decision maker/manager be willing to pay NOK 400.000 to have the project completed (estimated salary costs + overheads) after having read the abstract?  Why/why not?
\end{enumerate}

\end{abstract}


\tableofcontents

\chapter{LOLOLL}
\section{Dette er en subtittel}

\chapter{Contents of the project description}
The project description must use the gucmasterproject class file and contain the following elements/chapters:
\begin{itemize}
\item[] Front page
\item[] Table of contents
\item[] Abstract
\item[1] Introduction
\item[1.1] Topic covered
\item[1.2] Keywords
\item[1.3] Problem description
\item[1.4] Justification motivation and benefits
\item[1.5] Research questions
\item[1.6] Planned contributions
\item[2]Related work
\item[3] Choice of methods
\item[4]Milestones, deliverables and resources
\item[5]Feasibility study
\item[6]Risk analysis
\item[7]Ethical and legal considerations
\item[] Bibliography
\item[] Appendix
\item[A] Acronyms and abbreviations

\end{itemize}

Each chapter must contain the information specified in this document and further explained in 
lectures or included in lecture notes.

\chapter{Introduction (1-2 pages)}
The introduction chapter should not include detailed information on
how you intend to solve the problem, what you're going to do etc.
This belongs more in the 'method' and 'feasibility study' section
of the research proposal.

Make sure you read several of the past project proposals.
Make your own judgment on how 'good' they are.

\section{Topic covered by the project}
This section specifies the general area of the project.
It should preferably be understandable by everybody,
also those not familiar with the field. (e.g. all your relations and friends).

The purpose of the topic section is to:
\begin{itemize}
\item Very quickly give the reader some idea of the perspective taken
with respect to problem addressed.  
\item Help a reader to decide if the project
is within the readers area of interest and scope.
\item Help the author (you!) to see if he has the necessary skills,
if he/she needs to get access to specific expertise etc.
Do you have the right skills/ background/ knowledge
to be able to carry out the project?
\end{itemize}

\paragraph{Control questions:}
\begin{enumerate}
\item Does it have the right length?
\item Is it focused or is it just a non-focused brain dump going all over the place?
\item Is it clear from the text what skills would be required/beneficial in order to do/participate in the project?
\end{enumerate}


\section{Keywords}
There are several sources of keywords. Rather than 'inventing your own' you should select an appropriate set of keywords from a reputable source such as the one published by the IEEE computer society (IEEE Computer Society - Keywords).
%\protect\url{http://www.computer.org/portal/site/ieeecs/menuitem.c5efb9b8ade9096b8a9ca0108bcd45f3/index.jsp?&pName=ieeecs_level1&path=ieeecs/publications/author/keywords&file=ACMtaxonomy.xml&xsl=generic.xsl&},
The taxonomy by Avizienis et al\cite{Avizienis2004} provides an overview of of the subject area and an alternative set of keywords/classifications.

\paragraph{Control questions:}
\begin{enumerate}
\item Does the collection of keywords 'pin down' the project or is it to 'wide'?
\item Are the keywords too specific, making it difficult  for people with a closely related interest to recognize the keywords?
\item Why is it likely that a person working in the field would use the keywords you have selected when doing a search in this area?
\item Is the number of keywords appropriate?
\end{enumerate}

\section{Problem description}
What's 'wrong' with the world we're living in? E.g.
\begin{itemize}
\item   Something is currently too difficult.
\item   Something is broken/doesn't work properly.
\item   Something is currently to expensive, difficult, costly etc.
\end{itemize}

\paragraph{Control questions:}
\begin{enumerate}
\item Does it have an appropriate length?
\item Would it be possible to explain the problem description to a non-expert/expert in say 2 minutes in such a way that it was understood?
\item If explained to different people, would they have a common understanding?
\item If you were to check if your problem description was understood, what question(s) would you ask?
\item What is the information density of your text and why?
\end{enumerate}

\section{Justification, motivation and benefits}
This section should be understandable by everybody including your family and relatives.
In particular, it should be understandable to those who will benefit.
NOTE : 'I want to do zz' does not count as a legitimate motivation!
\begin{itemize}
\item Why is important to solve the problem you have identified?
\item Why would 'mankind' benefit from a solution to the problem identified?
\item Who would benefit (the stakeholders)?
\item What are the primary and secondary benefits - what's in it for the stakeholders?
\end{itemize}
You should try to find a journal, conference or newspaper article identifying the problem you will be adressing.
This can be used to substantiate your claim that the problem you are adressing is significant.

\paragraph{Control questions}
\begin{enumerate}
\item For each of the issues listed above, has the issue been addressed properly/thoroughly? 
\item What is the information density of your text and why?
\item If the project results was to be put in an auction when the project was completed - what price would it fetch and who would put in what bids? 
\item What would be the overall ROI (Return On Investment) of your project if carried out?
\end{enumerate}

\section{Research questions}\label{research:questions}
Describe the types of information you need in order to solve the research problem, e.g.
We need to find out
\begin{itemize}
\item what factors affect  xx (where xx is the 'parameter' you want to improve, e.g. cost, time, usability, security, etc.)
\item to what extent will activity/ method/procedure yy (where yy is some method of improving the parameter, e.g.  a program for simplifying access) improve factor xx?
\item have somebody solved this or some closely related problem?
\item how well has the problem been solved?
\item what is the theoretically 'best' one can achieve?
\end{itemize}

\paragraph{Control questions:}
\begin{enumerate}
\item Are there any questions at all? Look for '?'...
\item Why are the research questions relevant to the research problem?
\item What other research questions might also be relevant?
\item why/why not are the chosen research questions the most relevant?
\end{enumerate}

\section{Planned contributions}
A short summary of what kind of {\em new} results the master thesis will produce.  
Ideally,  the potential novelty of the results should be justified by means of references provided.
E.g. if an article describes the problem you will be adressing as {\em unsolved},
you should include this reference.  Similarly, if you e.g. have some ideas on how an 
authentication method can be improved in terms of FAR/FRR, you should specify the best
 FAR/FRR figures published and a reference to where this was published. 
 The goal of the master thesis
will be to produce the new results identified in this section.

\paragraph{Control questions}
\begin{enumerate}
\item Is the length of the section appropriate and why?
\item Why/why not are the contributions 'significant'?
\item Why/why not is it realistic that the planned contribution can be achieved?  You may want to have a look at relevant literature/ other completed master thesis to answer this question.
\end{enumerate}

\chapter{Related work (3-10 pages)}
The purpose of this chapter
is to explain to the reader what knowledge is already
available from the literature.

The purpose of the related work chapter is to:
\begin{itemize}
\item Identify to what extent information identified in the 'Research questions'  section is provided in the literature.
\item Give an overview of why/how the literature provides the answer to the research questions identified.
\item Identify areas/ research questions where the literature appears to be weak or non-existent.
\end{itemize}
The Related Work Chapter is NOT:
\begin{itemize}
\item   A list of abstracts and summaries of more-or-less-relevant literature.
\end{itemize}
If you have
\begin{itemize}
\item   found some relevant literature
\item   made summaries of what you have written
\end{itemize}
you should
\begin{itemize}
\item reorganize these summaries to focus on the research questions you have identified.
\end{itemize}

This chapter should include one subsection for each of the research
questions identified in section \ref{research:questions}.  

\section{Handling Potential problems}
When searching for literature, you usually get too many hits or none at all...

\paragraph{Question 1} I don't find any relevant literature.

\paragraph{Answer 1.A}  Make a list of words, phrases, applications, abbreviations,
organizations, terminology etc. relevant for your area of interest.
Ask a librarian to sit with you for 20 minutes to formulate relevant
queries to available databases.  Record your findings.

\paragraph{Answer 1.B}  Go to the ACM (www.acm.org) or IEEE (www.ieee.org) web pages.
Identify the SIGs (Special Interest Groups) of these organizations.
Select the SIGs which looks the most interesting.
Most SIGs publish one or more journals and/or organize workshops or conferences.
Get hold of a few journals or proceedings and see if they're any interesting.


\paragraph{Question 2}  I've found a lot of papers.
They all look interesting, but I don't have time to read them all.

\paragraph{Answer 2.A}  Narrow your search.  Be more specific in your search.  Read the abstracts of the relevant articles before you read the full papers.

\paragraph{Answer 2B}  Find a citation index (e.g. \url{http://citeseer.ist.psu.edu/}.
Read those papers with a high citation score first
(a citation index rates papers according to 'academic popularity').  Alternatively,
read those papers published in 'prestigious' conference proceedings or journals first.


\paragraph{Control questions:}
\begin{enumerate}
\item Why can we have confidence that the most relevant literature has been identified?
\item is the related literature grouped in a sensible way such that the reader gets a good understanding of 'existing knowledge' relating to th research questions/problem description?
\item Is the chapter sufficiently comprehensive?
\end{enumerate}

\chapter{Choice of methods (2-5 pages)}
This section is to include a description of the methods to be used,
including references to literature describing the methods to be used
(e.g. qualitative, quantitative, interviews, surveys,
questionnaire,  model building etc.)
For each of the research questions to be addressed,
the chapter is to explain why the method is
\begin{itemize}
\item appropriate
\item likely to provide the desired knowledge/information.
\end{itemize}

\chapter{Milestones, deliverables and resources (2-5 pages)}
The purpose of this chapter is to convince the reader that you know exactly what to do.
This chapter gives a description of how the project is to be
broken down into smaller parts and activities.
\begin{enumerate}
\item  What is it you have to do in order to obtain the desired knowledge?
\item  What deliverables are to be produced (MSc thesis report, software,...)
\item  When are the various deliverables going to be available?
\end{enumerate}

For each deliverable, identify 4 versions, having an
'increasing' degree of completeness/quality.
Students are strongly recommended to review each others drafts.
For each version of a deliverable explain why and how this version is to
be better/more complete.  E.g. v1.0: my first draft -
chapter text includes 1/2 page summaries only.
v.2.0: Like v1.0, but comments by NN(who? fellow student)
has been incorporated. v3.0:....

This section is to include a preliminary table of contents for the MSc thesis
(only include 2 levels).

For each of the activities identified, specify
\begin{enumerate}
\item  the time you need to complete each activity both calendar time and 'man-hours'.
\item  hours needed by you
\item  things you need to buy (consumables)
\item  equipment, lab space or facilities you need access to
\item  contributions from others (e.g. survey/interview participants) and how much each will have to contribute in terms of resources (probably time)
\end{enumerate}
At the beginning of this section, provide a 2-3 line summary of the
resource requirements.  This is particularly useful if you have broken
down the task into a lot of small tasks.

\chapter{Feasibility study (1/2-3 pages)}
An analysis of why it is likely that the desired
results can be produced within the given time and
resource bounds.  This may include a description of
\begin{itemize}
\item similar projects completed by others and their 'resource consumption',
\item an attempt to answer parts of the research questions
\item the 'difficult' elements of the work and an explanation of why/how these problems can be solved.  
Alternatively you can explain an 'approximate' solution.
\end{itemize}

\chapter{Risk analysis (1/2-2 pages)}
\begin{itemize}
\item What can possibly go wrong when you do your project?
\item How do you intend to reduce impact of/solve these problems?
\end{itemize}

\chapter{Ethical and legal considerations (1/4-1 page)}
The purpose of this chapter is to convince the
reader and your self that your project activities are
\begin{itemize}
\item legal
\item ethical, e.g. don't use/distribute/collect etc. data in such a way that individuals may suffer.
\end{itemize}
For example, if you are planning to do reverse engineering activities, surveys, PENTESTING- (both technical and based on social engineering techniques) you need to be particularly careful and check with the appropriate experts and authorities if the activity is permitted.  An explanation of why your project is both legal and ethical should be given in this chapter.

If you need permission (e.g. because you will be collecting or processing privacy related information), 
you should  include the appropriate applications/application forms and ensure that these applications 
are submitted well before you need the permission.

%9.  Bibliography (1-3 pages)

\chapter{The bibliography}
References/bibliography
Reference to other peoples work MUST include:
\begin{itemize}
\item Name of author(s)  or name of responsible organization (if the document does not have a named author.
\item Title of document.
\item Where published.
\item Year of publication.
\end{itemize}

There are many different types of documents (article, book etc.).
In many cases, the required/optional fields may differ.
See e.g. the BibTeX entry of wikipedia (\url{http://en.wikipedia.org/wiki/BibTeX}).
The bibliography file 
\verb+imt4441.bib+ 
contains an example.

NOTE: A url on its own is no good!
It must be possible to locate the reference even if a link goes dead!

\appendix
\chapter{Known problems with MikTeX and TeXnicCenter}
\section{Hig logo xor url's}
\paragraph{Problem}
Using MikTeX (version 2.6) in combination with TeXnicCenter you may experience the following problems:

You have to choose between being able to get the HIG logo on the front page (Use the TeXnic translation LaTeX => ps => pdf)  and getting visible url's (use the LaTeX => PDF translation option).

\paragraph{Possible solution 1}
use \verb+\usepackage[dvips]{graphicx}+ in combination with TeXnicCenter translation LaTeX => PS, and use Adobe distiller to convert from PS to PDF.  This should give both logo and URL (but no line breaks in URL's).

\paragraph{Possible solution 2}
Use \verb+epstopdf.exe+ to convert \verb+higlogo.eps+ to \verb+higlogo.pdf+.
Modify the file \verb+gucmasterthesis.cls+ to include \verb+higlogo.pdf+ instead of \verb+higlogo+.
Select the TeXnicCenter translation \verb+LaTeX => PDF+.


\section{\LaTeX/AFPL Ghostscript crashes}
\paragraph{Problem}
The build window prints out operand stack, execution and dictionary stack before concluding with the message 
'\verb+Unrecoverable error, exit code 1+'.
\paragraph{Possible solution}
Use the TeXnicCenter translation 'LaTeX => PDF'.

%\chapter{Project description evaluation criteria}


%\bibliographystyle{plain}
\bibliographystyle{gucmasterthesis}
\bibliography{imt4441}



\end{document}





IEEE computer society keywords
http://www.computer.org/portal/site/ieeecs/menuitem.c5efb9b8ade9096b8a9ca0108bcd45f3/index.jsp?&pName=ieeecs_level1&path=ieeecs/publications/author/keywords&file=ACMtaxonomy.xml&xsl=generic.xsl&
